%%%%%%%%%%%%%%%%%%%%%%%%%%%%%%%%%%%%%%%%%
% Medium Length Professional CV
% LaTeX Template
% Version 2.0 (8/5/13)
%
% This template has been downloaded from:
% http://www.LaTeXTemplates.com
%
% Original author:
% Trey Hunner (http://www.treyhunner.com/)
%
% Important note:
% This template requires the resume.cls file to be in the same directory as the
% .tex file. The resume.cls file provides the resume style used for structuring the
% document.
%
%%%%%%%%%%%%%%%%%%%%%%%%%%%%%%%%%%%%%%%%%

%----------------------------------------------------------------------------------------
%	PACKAGES AND OTHER DOCUMENT CONFIGURATIONS
%----------------------------------------------------------------------------------------

\documentclass[9pt]{resume} % Use the custom resume.cls style

\usepackage[left=0.75in,top=0.6in,right=0.75in,bottom=0.6in]{geometry} % Document margins
\usepackage[table, dvipsnames]{xcolor}

\name{Nathan P. Walter} % Your name
\address{Current Address:  120 Talbot Laboratory, 104 S. Wright St. \\ Urbana, Illinois 61801} % Your address
\address{Permanent Address:  2320 Thayer St. \\ Evanston, Illinois 60201} % Your secondary addess (optional)
\address{(847)~$\cdot$~849~$\cdot$~7896 \\ \color{blue}{walter9-(at)-illinois-(.)-edu}} % Your phone number and email


\begin{document}
%----------------------------------------------------------------------------------------
%	EDUCATION SECTION
%----------------------------------------------------------------------------------------

\begin{rSection}{Education}

{\bf University of Illinois at Champaign-Urbana (UIUC)} \hfill {08.2013 -- present} \\
{\it{PhD Candidate}} in Nuclear, Plasma, and Radiological Engineering (NPRE) \hfill{Expected: 08.2018}\\
{\it{Master of Science}} in Nuclear, Plasma, and Radiological Engineering (NPRE) \hfill{08.2018}\\
{\it{Graduate Minor}} in Computational Science and Engineering\\
Advisor: Yang Zhang
\begin{itemize}
	\item Master's Thesis Topic: Direct Energy Landscape Sampling of the Homogeneous Nucleation and Crystal Growth of a Model Liquid
\end{itemize}

{\bf University of Illinois at Champaign-Urbana (UIUC)} \hfill {08.2010 -- 12.2014} \\ 
{\it Bachelor of Science} in Nuclear, Plasma, and Radiological Engineering (NPRE) \\
Minor in Mathematics \hfill{Overall GPA: 3.84/4.00}
\end{rSection}

\begin{rSection}{Research Interests}
	Understanding slow material processes from a atomistic scale; Rare event sampling methods such as metadynamics; Neutron and X-ray scattering; Classical and Ab Initio molecular dynamics for modeling and simulation; Materials undergoing irradiation; Large deformation constitutive material equations. Machine Learning algorithms
\end{rSection}

\begin{rSection}{Awards, Honors, Clubs, and Certificates}
	American Physical Society, GSOFT Travel Award \hfill{03.2017}
	\\[3pt]
	Graduate Specialization in Computational Science and Engineering \hfill{08.2016}
	\\[3pt]
	U.S. Department of Energy, Naval Reactors (NR), {\it Rickover Fellowship Program} Honorable Mention \hfill{00.2014} 
	\\[3pt]
	{ Nuclear Regulatory Commission Undergraduate Scholarship} \hfill{12.2011 -- 06.2013}
	\\[3pt]
	University of Illinois at Champaign-Urbana Dean's List Recipient \hfill{06.2011 -- 0.62013}
	\\[3pt]
	{\it{The Hacker Within}}, An organization for computational scientists to share and practice computational skills.\\
	\strut\hfill {Member:   08.2015 -- present}\\
	\strut\hfill Treasure: 08.2016 -- present
\end{rSection}

%----------------------------------------------------------------------------------------
%	TECHNICAL STRENGTHS SECTION
%----------------------------------------------------------------------------------------

\begin{rSection}{Technical Strengths}
	\begin{tabular}{ @{} >{\bfseries}l @{\hspace{6ex}} l }
		Computer Programming Languages &  C, C++, Matlab, Python, Fortran, Java, \LaTeX, Swift (novice),\\ & AJAX, R, OpenMP, MPI, HTML, CSS, Julia (novice)
		\\[5pt]
		Software & GROMACS, LAMMPS, VASP, SRIM/TRIM, FLAG, \\ & VMD, IGOR Pro, Dave, gnuplot, Adobe Photoshop, \\ & Illustrator, Flash, SPSS
	\end{tabular}
\end{rSection}

%----------------------------------------------------------------------------------------
%	WORK EXPERIENCE SECTION
%----------------------------------------------------------------------------------------

\begin{rSection}{Research Experience}
	{\bf PhD Research} \hfill {08.2016 -- present}
	\begin{itemize}\setlength\itemsep{-3pt}
		\item Implemented a method of directly sampling the energy landscape into the molecular dynamics package GROMACS in order to study the activation barrier statistics of various protein systems 
		\item Used energy landscape sampling to provide insight to protein folding and unfolding dynamics
		\item Developed a new method of directly sampling the energy landscape with higher computational efficiency than the one implemented in my master's work.
		\item Sampled and studied the energy landscapes of vary potentials to determine the affect of potential softness on the landscape
	\end{itemize}
	{\centering\noindent\rule{5cm}{0.4pt}}
	\\
	{\bf Master's Degree Research} \hfill {01.2014 -- 08.2016}
	\begin{itemize}\setlength\itemsep{-3pt}
		\item Implemented a method of directly sampling the energy landscape into the molecular dynamics package GROMACS in order to study the activation barrier statistics of glass forming and crystal forming systems
		\item Developed reduction codes to extract quantities from classical and \textit{ab initio} molecular dynamics simulations relevant for comparing simulations to scattering experiments (i.e. intermediate scattering function, density of states, etc.).
		\item Developed an open-source package, \textit{LiquidLib}, to analyze molecular dynamics trajectories to study the structure and dynamics of liquids and compare the results to neutron scattering experiments 
		\item Performed \textit{ab initio} molecular dynamic simulations to study the vibrational modes in D$_2$O and compare to neutron scattering experiments conducted at SEQUOIA, SNS, ORNL. 
		\item Performed \textit{ab initio} molecular dynamic simulations to study the effects of hydrogen impurities on liquid lithium transport properties
		\item Created a high dimensional molecular dynamics package to study the dimensionality of various quantities
	\end{itemize}
	{\centering\noindent\rule{5cm}{0.4pt}}
	\\
	{\bf Neutron and X-ray Scattering Summer School} \hfill {06.2015}
	\begin{itemize}\setlength\itemsep{-3pt}
		\item Studied x-ray scattering methods at the Advanced Photon Source, APS, Argonne National Laboratory (ANL)
		\item Studied neutron scattering methods at SNS and HFIR, Oak Ridge National Laboratory (ORNL)
	\end{itemize}
	{\centering\noindent\rule{5cm}{0.4pt}}
	\\
	{\bf Scattering Experiments}
	\begin{itemize}\setlength\itemsep{-3pt}
		\item Participated in pair distribution experiments on glass forming metallic liquids using a neutron electrostatic levatator performed at NOMAD, SNS, Oak Ridge National Laboratory (ORNL)
		\item Participated on Inelastic Neutron Scattering experiments on liquid metals performed at CNCS, SNS, Oak Ridge National Labratory (ORNL) 
		\item Analyzed scattering data on D$_2$O performed at SEQUOIA, SNS, Oak Ridge National Laboratory (ORNL)
	\end{itemize}
	{\centering\noindent\rule{5cm}{0.4pt}}
	\\
	{\bf Los Alamos Computational Physics Student Summer Workshop} \hfill {Summer 2014}
	\begin{itemize}\setlength\itemsep{-3pt}
		\item Implemented a strain-based constituent equation for large material deformation under high strain-rates into a production hydrocode
		\item Developed concepts for extending the strain-based formulation from perfectly plastic materials to rate-hardening materials.
		\item Studied the advantages of the strain-based with pertaining to advection in Lagrangian mode, finite material rotations, and artificial viscosity.
	\end{itemize}
	{\centering\noindent\rule{5cm}{0.4pt}}
	\\
	{\bf Machine Learning Experience}
	\begin{itemize}\setlength\itemsep{-3pt}
		\item Enrolled in several high level statistics courses, including the course on machine learning
		\item Participated in the Kaggle competition for Springleaf as a team. 
		\item For the competition, used various machine learning methods to reduce the data space, and build predictive models
		\item Used several machine learning regression and clustering methods to create a model to predict the value of a hand written input number
	\end{itemize}
	{\centering\noindent\rule{5cm}{0.4pt}}
	\\
	{\bf Institute for Genomic Biology} \hfill {Summer 2012}\\
	Undergraduate Research Assistant to Biofuel Lab Research \hfill {Champaign, IL}
	\begin{itemize}\setlength\itemsep{-3pt}
		\item Worked on British Petroleum (BP) Biofuel Project
		\item Analyzed soil samples for carbon/nitrogen make-up
		\item Studied different plants' potential as a biofuel
	\end{itemize}
	{\centering\noindent\rule{5cm}{0.4pt}}
	\\
	{\bf University of Northeastern Illinois} \hfill {Summer 2009}\\
	Student Research Assistant on Abstract Topology Project \hfill {Chicago, IL}
	\begin{itemize}\setlength\itemsep{-3pt}
		\item Implemented Java code to simulate contact points
		\item Developed mathematical and programming algorithms for the project
	\end{itemize}
	
\end{rSection}

\begin{rSection}{Presentations}
	Talk, American Physical Society March Meeting, {\it{``Protein Folding and Unfolding Dynamics from Direct Energy Landscape Sampling Simulations''}} \hfill{03.2017}
	\\[5pt]
	Talk, University of Illinois Urbana-Champaign Nuclear Engineering Graduate Seminar, {\it{``Direct Energy Landscape Sampling of the Homogeneous Nucleation and Crystal Growth of a Model Liquid''}} \hfill {12.2016}
	\\[5pt]
	Discussion, The Hacker Within: University of Illinois Urbana-Champaign, {\it{``An Overview of Techniques and Methods in Machine Learning with Application to Sci-Kit (sklearn) in Python''}} \hfill {11.2016}
	\\[5pt]
	Talk, University of Illinois Urbana-Champaign Soft Materials Seminar, {\it{``Direct Energy Landscape Sampling of the Homogeneous Nucleation and Crystal Growth of a Model Liquid''}} \hfill {09.2016}
	\\[5pt]
	Talk, American Conference on Neutron Scattering, {\it{``Homogeneous Nucleation and Crystal Growth in a Model Liquid from Direct Energy Landscape Sampling Simulations''}} \hfill {07.2016}
	\\[5pt]
	Discussion, The Hacker Within: University of Illinois Urbana-Champaign, {\it{``Understanding Classification of Hand-Written Numbers with Machine Learning Techniques''}} \hfill {05.2016}
	\\[5pt]
	Talk, University of Illinois Urbana-Champaign Nuclear Engineering Undergraduate Seminar, {\it{``Homogeneous Nucleation and Crystal Growth in a Model Liquid from Direct Energy Landscape Sampling Simulations''}} \hfill {04.2016} 
	\\[5pt]
	Talk, American Physical Society March Meeting, {\it{``Homogeneous Nucleation and Crystal Growth in a Model Liquid from Direct Energy Landscape Sampling Simulations''}} \hfill {03.2016}
	\\[5pt]
	Poster, American Physical Society March Meeting, {\it{``Energy Landscape Statistics of Kob-Andersen Liquid From Direct Energy Barrier Sampling''}} \hfill {03.2015}
	\\[5pt]
	Talk, Los Alamos Student Summer Symposium, {\it{``A New Strain-Based Method for Plastic Flow Simulations''}} \hfill {08.2014} \\
\end{rSection}

\begin{rSection}{Publications}
	T. P. Moneypenny, {\bf N. P. Walter}, Z. Cai, Y. Miao, D. L. Gray, J. J. Hinman, S. Lee, Y. Zhang, J. S. Moore, {\it { ``Impact of Shape Persistence on the Porosity of Molecular Cages''} } JACS (2017)
	\\[5pt]
	{\bf Nathan Walter}, Zhikun Cai, Abhishek Jaiswal, Yang Zhang, {\it ``LiquidLib: A comprehensive tool for post processing of classical and \textit{ab initio} molecular dynamics simulations of liquids with application to neutron scattering experiments''}, to be submitted
	\\[5pt]
	Zhikun Cai, Abhishek Jaiswal, {\bf Nathan Walter}, Yang Zhang, {\it ``Validity boundary of the Stokes-Einstein relation in water''} to be submitted
	\\[5pt]
	{\bf Nathan Walter}, Yang Zhang, {\it ``Direct Energy Landscape Sampling of the Homogeneous Nucleation and Crystal Growth of a Model Liquid''}, to be submitted
	\\[5pt]
	Zhikun Cai, {\bf Nathan Walter}, Yang Zhang, {\it ``Energy Landscape Statistics And Coarsening In Liquids: A Relaxation Excitation Mode Analysis: Part I}", to be submitted.
	\\[5pt]
	Zhikun Cai, {\bf Nathan Walter}, Yang Zhang, {\it ``Energy Landscape Statistics And Coarsening In Liquids: A Relaxation Excitation Mode Analysis: Part II}", to be submitted.
	\\[5pt]
	{\bf Nathan Walter}, Zhikun Cai, Yang Zhang, {\it ``Energy Landscape Statistics And Coarsening In Liquids: A Relaxation Excitation Mode Analysis: Part III}", to be submitted.
	\\[5pt]
	{ \bf Nathan Walter}, Paul Friedrichsen, Scott Runnels, {\it{``Extending a Strain Space Formulation for Plasticity to Rate-Hardening Materials and Finite Rotations"}}, submitted to Mathematics and Computers in Simulation .
	\\[5pt]
	{ \bf Nathan Walter}, Paul Friedrichsen, Scott Runnels, {\it{``Extending a Strain Space Formulation for Plasticity to Rate-Hardening Materials and Finite Rotations"}}, LA-UR-15-23329, Los Alamos Unlimited Release (2015).
	\\[5pt]
	{ \bf Nathan Walter}, Paul Friedrichsen, {\it{``Improving Plasticity Modeling in Hydrocodes with Hypoelastic Frameworks"}}, LA-UR-14-26946, Los Alamos Unlimited Release (2014).
	\\	
\end{rSection}

\begin{rSection}{Appointments}
	{\bf Research Assistant} Yang Zhang's Research Group \hfill {January, 2014 -- present} 
	\\
	{\bf Nuclear Regulatory Commission Graduate Fellowship} \hfill {January 2014 -- present}
	\\
	{\bf Teaching Assistant}, NPRE 448: Nuclear Systems Engineering and Design\hfill {August, 2013 -- January, 2014}
\end{rSection}


\end{document}
