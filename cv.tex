%%%%%%%%%%%%%%%%%%%%%%%%%%%%%%%%%%%%%%%%%
% Medium Length Professional CV
% LaTeX Template
% Version 2.0 (8/5/13)
%
% This template has been downloaded from:
% http://www.LaTeXTemplates.com
%
% Original author:
% Trey Hunner (http://www.treyhunner.com/)
%
% Important note:
% This template requires the resume.cls file to be in the same directory as the
% .tex file. The resume.cls file provides the resume style used for structuring the
% document.
%
%%%%%%%%%%%%%%%%%%%%%%%%%%%%%%%%%%%%%%%%%

%----------------------------------------------------------------------------------------
%	PACKAGES AND OTHER DOCUMENT CONFIGURATIONS
%----------------------------------------------------------------------------------------

\documentclass{resume} % Use the custom resume.cls style

\usepackage[left=0.75in,top=0.6in,right=0.75in,bottom=0.6in]{geometry} % Document margins
\usepackage[table, dvipsnames]{xcolor}



\name{Nathan Walter} % Your name
\address{Current Address:  120 Talbot Laboratory, 104 S. Wright St. \\ Urbana, Illinois 61801} % Your address
\address{Permanent Address:  2320 Thayer St. \\ Evanston, Illinois 60201} % Your secondary addess (optional)
\address{(847)~$\cdot$~849~$\cdot$~7896 \\ \color{blue}{walter9@illinois.edu}} % Your phone number and email


\begin{document}
%----------------------------------------------------------------------------------------
%	EDUCATION SECTION
%----------------------------------------------------------------------------------------

\begin{rSection}{Education}

{\bf University of Illinois at Champaign-Urbana (UIUC)} \hfill {August, 2013 -- present} \\
{\it{Master of Science}} in Nuclear, Plasma, and Radiological Engineering (NPRE) \\
{\it{PhD Candidate}} in Nuclear, Plasma, and Radiological Engineering (NPRE) \\
Computational Science and Engineering Certificate \\
Expected Master's Degree Completion: December, 2015 \\
Expected PhD Completion: May, 2018\\
\\
Advisor: Yang Zhang
\\
\\
{\bf University of Illinois at Champaign-Urbana (UIUC)} \hfill { August, 2010 -- January, 2014} \\ 
{\it Bachelor of Science} in Nuclear, Plasma, and Radiological Engineering (NPRE) \\
Minor in Mathematics \\
Overall GPA: 3.84/4.00
\smallskip \\


\end{rSection}

\begin{rSection}{Research Interests}
	Understanding slow material processes from a atomistic scale; Neutron and X-ray scattering; Classical and Ab Initio molecular dynamics for modeling and simulation; Materials undergoing irradiation; Large deformation constitutive material equations. Machine Learning algorithms
	\\
\end{rSection}


\begin{rSection}{Appointments}
	{\bf Research Assistant} Yang Zhang's Research Group \hfill {January, 2014 -- present} 
	\\
	\\
	{\bf Nuclear Regulatory Commission Graduate Fellowship} \hfill {January 2014 -- present}
	\\
	\\
	{\bf Teaching Assistant} \hfill {August, 2013 -- January, 2014} \\
	NPRE 448: Nuclear Systems Engineering and Design
	\\
\end{rSection}


%----------------------------------------------------------------------------------------
%	WORK EXPERIENCE SECTION
%----------------------------------------------------------------------------------------


\begin{rSection}{Research Experience}
	{\bf Master's Degree Research} \hfill {January 2014 -- present}
	\begin{itemize}
	\item Implemented a method of directly sampling the energy landscape into the molecular dynamics package GROMACS in order to study the activation barrier statistics of various systems
	\item Developed reduction codes to extract quantities from classical and ab initio molecular dynamics simulations relevant for comparing simulations to scattering experiments (i.e. intermediate scattering function, density of states, etc.).
	\item Developed an open-source package, \textit{LiquidLib}, to analyze molecular dynamics trajectories to study the structure and dynamics of liquids and compare the results to neutron scattering experiments 
	\item Performed ab initio molecular dynamic simulations to study the vibrational modes in D$_2$O and compare to neutron scattering experiments conducted at SEQUOIA, SNS, ORNL. 
	\item Performed ab initio molecular dynamic simulations to study the effects of hydrogen impurities on liquid lithium transport properties
	\item Created a high dimensional molecular dynamics package to study the dimensionality of various quantities
	\end{itemize}
	{\centering\noindent\rule{5cm}{0.4pt}}
	\\
	{\bf Machine Learning Experience} \hfill {Fall 2015}
	\begin{itemize}
		\item Enrolled in several high level statistics courses, including the course on machine learning
		\item Participated in the Kaggle competition for Springleaf as a team. 
		\item For the competition, used various machine learning methods to reduce the data space, and build predictive models
	\end{itemize}
	{\centering\noindent\rule{5cm}{0.4pt}}
	\\
	{\bf Neutron and X-ray Scattering Summer School} \hfill {June 2015}
	\begin{itemize}
	\item Studied x-ray scattering methods at the Advanced Photon Source, APS, Argonne National Laboratory (ANL)
	\item Studied neutron scattering methods at SNS and HFIR, Oak Ridge National Laboratory (ORNL)
	\end{itemize}
	{\centering\noindent\rule{5cm}{0.4pt}}
	\\
	{\bf Scattering Experiments} \hfill {May 2014}
	\begin{itemize}
	\item Participated on Inelastic Neutron Scattering experiments on liquid metals performed at CNCS, SNS, Oak Ridge National Labratory (ORNL) 
	\item Analyzed scattering data on D$_2$O performed at SEQUOIA, SNS, Oak Ridge National Labratory (ORNL)
	\end{itemize}
	{\centering\noindent\rule{5cm}{0.4pt}}
	\\
	{\bf Los Alamos Computational Physics Student Summer Workshop} \hfill {Summer 2014}
	\begin{itemize}
	\item Implemented a strain-based constituent equation for large material deformation under high strain-rates into a production hydrocode
	\item Developed concepts for extending the strain-based formulation from perfectly plastic materials to rate-hardening materials.
	\item Studied the advantages of the strain-based with pertaining to advection in Lagrangian mode, finite material rotations, and artificial viscosity.
	\end{itemize}
	{\centering\noindent\rule{5cm}{0.4pt}}
	\\
	{\bf Institute for Genomic Biology} \hfill {Summer 2012}\\
	Undergraduate Research Assistant to Biofuel Lab Research \hfill {Champaign, IL}
	\begin{itemize}
	\item Worked on British Petroleum (BP) Biofuel Project
	\item Analyzed soil samples for carbon/nitrogen make-up
	\item Studied different plants' potential as a biofuel
	\end{itemize}
	{\centering\noindent\rule{5cm}{0.4pt}}
	\\
	{\bf University of Northeastern Illinois} \hfill {Summer 2009}\\
	Student Research Assistant on Abstract Topology Project \hfill {Chicago, IL}
	\begin{itemize}
	\item Implemented Java code to simulate contact points
	\item Developed mathematical and programming algorithms for the project
	\end{itemize}
	
\end{rSection}



%----------------------------------------------------------------------------------------
%	TECHNICAL STRENGTHS SECTION
%----------------------------------------------------------------------------------------

\begin{rSection}{Technical Strengths}

\begin{tabular}{ @{} >{\bfseries}l @{\hspace{6ex}} l }
	Computer Programming Languages &  C, C++, Matlab, Python, Fortran, Java, \LaTeX, Swift,\\ & AJAX, R, OpenMP, MPI, HTML, Julia (novice)\\
	\\
	Software & GROMACS, LAMMPS, VASP, SRIM/TRIM, FLAG, \\ & VMD, IGOR Pro, Dave, gnuplot, Adobe Photoshop, \\ & Illustrator, Flash, SPSS

\end{tabular}
\\

\end{rSection}

\begin{rSection}{Publications}
	{ \bf Nathan Walter}, Paul Friedrichsen, Scott Runnels, {\it{``Extending a Strain Space Formulation for Plasticity to Rate-Hardening Materials and Finite Rotations"}}, submitted to Mathematics and Computers in Simulation .
	\\
	\\
	{ \bf Nathan Walter}, Paul Friedrichsen, Scott Runnels, {\it{``Extending a Strain Space Formulation for Plasticity to Rate-Hardening Materials and Finite Rotations"}}, LA-UR-15-23329, Los Alamos Unlimited Release (2015).
	\\
	\\
	Zhikun Cai, {\bf Nathan Walter}, Yang Zhang, {\it ``Energy Landscape Statistics And Coarsening In Liquids: A Relaxation Mode Analysis}", to be submitted.
	\\
	\\
	{ \bf Nathan Walter}, Paul Friedrichsen, {\it{``Improving Plasticity Modeling in Hydrocodes with Hypoelastic Frameworks"}}, LA-UR-14-26946, Los Alamos Unlimited Release (2014).
	\\	
\end{rSection}

\begin{rSection}{Presentations}
	Talk, American Physical Society March Meeting, {\it{``Homogenous Nucleation and Crystal Growth in a Model Liquid from Direct Energy Landscape Sampling Simulations"}} \hfill {March, 2016} \\
	\\
	Poster, American Physical Society March Meeting, {\it{``Energy Landscape Statistics of Kob-Andersen Liquid From Direct Energy Barrier Sampling''}} \hfill {March, 2015} \\
	\\	
	Contributed Talk, Los Alamos Student Summer Symposium, {\it{``A New Strain-Based Method for Plastic Flow Simulations''}} \hfill {August, 2014} \\
\end{rSection}


\begin{rSection}{Awards and Honors}
U.S. Department of Energy, Naval Reactors (NR), {\it Rickover Fellowship Program in Nuclear Engineering} Honorable Mention \hfill {April, 2014} 
\\
\\
{ Nuclear Regulatory Commission Undergraduate Scholarship} \hfill {Fall, 2011 -- Spring, 2013}
\\
\\
University of Illinois at Champaign-Urbana Dean's List Recipient \hfill{Spring 2011 -- Spring 2013}
\end{rSection}

%----------------------------------------------------------------------------------------
%	EXAMPLE SECTION
%----------------------------------------------------------------------------------------

%\begin{rSection}{Section Name}

%Section content\ldots

%\end{rSection}

%----------------------------------------------------------------------------------------

\end{document}
